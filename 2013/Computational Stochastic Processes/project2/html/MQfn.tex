
% This LaTeX was auto-generated from an M-file by MATLAB.
% To make changes, update the M-file and republish this document.

\documentclass{article}
\usepackage{graphicx}
\usepackage{color}

\sloppy
\definecolor{lightgray}{gray}{0.5}
\setlength{\parindent}{0pt}

\begin{document}

    
    \begin{verbatim}
function [ mErr ] = MQfn( Tmax,dt,mu )
numRuns = 100;
T = linspace(0,Tmax,Tmax/dt);
dW = randn(numRuns,length(T));
X = zeros(numRuns,length(T));
lambda = -1.0;
theta = 0.5;

% %EM scheme
% for i = 1:numRuns
%    X_j = 1.1;
%    X(i,1) = X_j;
%    for j = 2:length(T)
%       X_j = X_j - lambda*X_j*(1-X_j)*dt - mu*X_j*(1-X_j)*sqrt(dt)*dW(i,j);
%       X(i,j) = X_j;
%    end
% end

% %Milstein scheme
% for i = 1:numRuns
%    X_j = 1.1;
%    X(i,1) = X_j;
%    for j = 2:length(T)
%       X_j = X_j - lambda*X_j*(1-X_j)*dt - mu*X_j*(1-X_j)*sqrt(dt)*dW(i,j) - 0.5*(mu*mu*X_j*(1-X_j) - 2*mu*mu*X_j*X_j*(1-X_j));
%       X(i,j) = X_j;
%    end
% end

% semi-implicit Milstein scheme
for i = 1:numRuns
   X_j = 1.1;
   X(i,1) = X_j;
   for j = 2:length(T)
       % use anonymous function to define newton-raphson root quickly
       milstein = @(x) x*(1+theta*lambda*dt*(1-x)) - (X_j - lambda*X_j*(1-X_j)*dt - mu*X_j*(1-X_j)*sqrt(dt)*dW(i,j) - 0.5*(mu*mu*X_j*(1-X_j) - 2*mu*mu*X_j*X_j*(1-X_j)));
       x = fzero(milstein, X_j);
       X_j = x;
       X(i,j) = X_j;
   end
end

err = (X-1);
err = err.^2;
mErr = mean(err);

end
\end{verbatim}

        \color{lightgray} \begin{verbatim}Error using MQfn (line 3)
Not enough input arguments.
\end{verbatim} \color{black}
    


\end{document}
    
